\documentclass[12pt]{article}
\usepackage[utf8]{inputenc}
\usepackage{ulem}
\usepackage[english,russian]{babel}
\usepackage{amssymb}
\usepackage{graphicx}
\usepackage{setspace}
\usepackage{geometry}
\usepackage[english]{blindtext}
\usepackage[matrix,arrow,curve,frame,poly,arc]{xy}
\usepackage{fancyhdr}
\usepackage{subcaption}
\usepackage[export]{adjustbox}
\usepackage{wrapfig}
\usepackage{amsmath}
\usepackage{subfig}
\graphicspath{ {image} }
\usepackage{hyperref}
\hypersetup{pdfstartview=FitH,  linkcolor=linkcolor,urlcolor=urlcolor, colorlinks=true}

\begin{document}

    \begin{titlepage}
        \thispagestyle{empty}
        \begin{center}
            \noindent\begin{minipage}{0.14\textwidth}
                         \includegraphics[width=\linewidth]{image/madi_logo}
            \end{minipage}%
            \begin{minipage}{0.86\textwidth}
                \center{\small{\vspace{\baselineskip}}}
                \center{{МИНИСТЕРСТВО НАУКИ И ВЫСШЕГО ОБРАЗОВАНИЯ РОССИЙСКОЙ ФЕДЕРАЦИИ \\ федеральное бюджетное образовательное государственное учреждение высшего образования}
                    \center{{\hspace{1 cm}}}}
            \end{minipage}
            \small{\textbf{<<МОСКОВСКИЙ АВТОМОБИЛЬНО-ДОРОЖНЫЙ ГОСУДАРСТВЕННЫЙ ТЕХНИЧЕСКИЙ УНИВЕРСИТЕТ (МАДИ)>>}}\\
            \vspace{0.2 cm}
            \scriptsize{{\textbf{КАФЕДРА <<ВЫСШАЯ МАТЕМАТИКА>> }}}
            \vspace{\baselineskip}

            \small{\textbf{Реферат}}\\
            \vspace{0.2 cm}

            по дисциплине <<Базы данных>>
            на тему\\
            <<Алгоритм ChaCha20>>
        \end{center}

        \vspace{5cm}
        \hfill \begin{minipage}{0.5\linewidth}
                   \raggedleft
                   \textbf{Выполнил:}\\
                   Учебная группа 2бПМ\\
                   Осада.В.В\\
                   \textbf{Принял:}\\
                   Лебедев А.Н\\
                   Подпись \underline{\hspace{1cm}}\\
        \end{minipage}

        \vfill

        \begin{center}
            Москва 2023
        \end{center}
    \end{titlepage}

    \tableofcontents
    \newpage


    \section{Введение}

    В современном цифровом мире защита информации является одной из ключевых задач, особенно при передаче данных через открытые сети.
    Шифрование данных стало неотъемлемой частью безопасности в онлайн-коммуникациях, и для этой цели разработано множество алгоритмов.
    Однако, помимо надежности, важным фактором является эффективность работы таких алгоритмов, особенно в контексте мобильных устройств и ограниченных вычислительных ресурсов.

    В этом реферате мы сосредоточимся на рассмотрении ChaCha20 - легковесного алгоритма шифрования, разработанного Дэниэлем Дж. Бернштейном (Daniel J. Bernstein).
    ChaCha20 был создан как альтернатива более распространенным алгоритмам, таким как AES (Advanced Encryption Standard), и изначально предназначался для использования в программном обеспечении и на устройствах с ограниченными вычислительными ресурсами.

    ChaCha20 является симметричным алгоритмом шифрования, который обеспечивает безопасность и высокую производительность.
    Он основан на операциях XOR и сложении по модулю 2 и использует 32-битные слова для обработки данных.
    Алгоритм ChaCha20 также предоставляет возможность генерации различных ключей для каждого блока данных с помощью счетчика.
    Благодаря своей простоте и эффективности, ChaCha20 нашел широкое применение в различных областях, включая криптографию, защиту данных в компьютерных сетях и приложениях реального времени.

    В данном исследовании мы проанализируем основные принципы работы и характеристики алгоритма ChaCha20, а также рассмотрим его преимущества и ограничения.
    Также мы рассмотрим некоторые примеры применения ChaCha20 в реальных сценариях, где этот алгоритм проявляет свои преимущества.

    \newpage


    \section{ChaCha20}

    \subsection{История создания}
    Он был разработан криптографом Дэниэлем Йенсеном в 2008 году и представляет собой модифицированную версию алгоритма Salsa20.

    История создания алгоритма ChaCha20 началась с появления Salsa20, который был разработан Дэниэлем Бернштейном в 2005 году.
    Salsa20 был представлен как быстрый и безопасный потоковый шифр, обладающий хорошей производительностью на различных платформах.
    Однако, когда в 2008 году исследователи обнаружили некоторые уязвимости в Salsa20, Дэниэль Бернштейн решил создать его улучшенную версию.

    ChaCha20 был разработан Бернштейном как альтернатива алгоритму Salsa20, с целью исправить некоторые недостатки и уязвимости предыдущего алгоритма.
    Бернштейн использовал аналогичный подход, основанный на операциях XOR и сложении по модулю 2.
    Однако, он внес некоторые изменения в структуру алгоритма, чтобы обеспечить большую стойкость и улучшить производительность.

    Основной идеей ChaCha20 является использование 32-битных слов и операций сложения по модулю $2^{32}$, которые выполняются над этими словами.
    Алгоритм также использует криптографические ключи и счетчик, который увеличивается с каждым блоком данных, чтобы генерировать различные ключи для каждого блока.
    Это обеспечивает высокую степень безопасности и защиты от атак.

    ChaCha20 быстро получил популярность среди разработчиков и криптографов благодаря своей простоте, безопасности и хорошей производительности.
    Он стал широко использоваться в различных приложениях и протоколах, включая шифрование в сети Интернет, мобильные устройства и криптовалюты.

    В 2014 году ChaCha20 был выбран как один из шифров в стандарте Internet Engineering Task Force (IETF) для шифрования в протоколе TLS (Transport Layer Security).
    Это дало ему дополнительный признак надежности и подтверждение его статуса надежного алгоритма.

    \subsection{О работе алгоритма}
    Алгоритм ChaCha20 основан на концепции потокового шифра, который генерирует псевдослучайную последовательность байтов, называемых ключевым потоком.
    Этот ключевой поток затем используется для шифрования и дешифрования данных.
    Основным преимуществом ChaCha20 является его высокая скорость работы, которая обусловлена эффективным использованием операций сложения и побитового исключающего ИЛИ (XOR).

    \textbf{Алгоритм ChaCha20 состоит из следующих шагов:}

    \begin{enumerate}
        \item \textbf{Инициализация}: Задается 256-битный ключ и 64-битный счетчик, который увеличивается с каждым новым блоком данных.
        Также может быть задан 64-битный случайный нонс (инициализирующий вектор).
        \item \textbf{Формирование блоков}: ChaCha20 разделяет входные данные на блоки размером 64 байта.
        \item \textbf{Генерация ключевого потока}: Каждый блок данных подвергается серии раундовых операций, включающих сложение, побитовое исключающее ИЛИ и перемешивание данных.
        Это приводит к формированию 512-битного ключевого потока.
        \item \textbf{Шифрование/дешифрование}: Исходные данные XOR-ятся с ключевым потоком для шифрования или дешифрования информации.
    \end{enumerate}

    \textbf{Применение:}

    \begin{itemize}
        \item \textbf{Криптография веба}: ChaCha20 используется в протоколе HTTPS вместе с алгоритмом шифрования Poly1305 для обеспечения безопасной передачи данных между клиентом и сервером.
        \item \textbf{Криптография мобильных устройств}: Благодаря своей высокой производительности и низкому потреблению энергии, ChaCha20 широко применяется в мобильных устройствах для шифрования коммуникации, файлов и данных приложений.
        \item \textbf{Виртуальные частные сети (VPN)}: Многие VPN-сервисы используют ChaCha20 в сочетании с протоколом обмена ключами WireGuard для обеспечения безопасной передачи данных через открытые сети.
        \item \textbf{Шифрование дисков}: ChaCha20 может быть использован для шифрования данных на жестком диске или других носителях информации, обеспечивая их конфиденциальность и защиту от несанкционированного доступа.
    \end{itemize}

    ChaCha20 является одним из ведущих легковесных алгоритмов шифрования, обеспечивающим сочетание безопасности и производительности, что делает его популярным выбором во многих приложениях, требующих защиты информации.

    \subsection{Пример использования алгоритма ChaCha20 в протоколе WireGuard VPN}
    Давайте рассмотрим конкретный пример использования алгоритма ChaCha20 в сочетании с протоколами обмена ключами WireGuard в VPN .

    WireGuard является современным протоколом VPN, который использует асимметричные криптографические ключи для установления безопасного соединения между клиентом и сервером.
    Каждая сторона генерирует пару ключей: приватный и публичный ключи.

    Процесс установления соединения в WireGuard включает в себя следующие шаги:

    \begin{enumerate}
        \item Генерация ключей:
        \begin{itemize}
            \item Клиент и сервер генерируют свои приватные и публичные ключи с помощью алгоритма эллиптической кривой (например, Curve25519). Приватный ключ является секретным и должен быть известен только соответствующей стороне, а публичный ключ может быть распространен.
        \end{itemize}

        \item Обмен публичными ключами:
        \begin{itemize}
            \item Клиент отправляет свой публичный ключ серверу, и наоборот. Обмен ключами может происходить через незащищенные каналы, так как публичные ключи не содержат конфиденциальной информации.
        \end{itemize}

        \item Обмен и проверка маркеров проверки подлинности (Handshake):
        \begin{itemize}
            \item Клиент и сервер обмениваются сообщениями, содержащими маркеры проверки подлинности и информацию о своих настройках безопасности.
            \item Оба участника выполняют проверку подлинности публичных ключей друг друга, чтобы убедиться, что общение происходит с ожидаемой стороной.
        \end{itemize}

        \item Генерация общего секрета:
        \begin{itemize}
            \item Используя свой приватный ключ и публичный ключ другой стороны, каждая сторона вычисляет общий секрет — общий секретный ключ, который будет использоваться для шифрования и расшифрования данных в дальнейшем.
        \end{itemize}

        \item Формирование ключей шифрования с помощью ChaCha20:
        \begin{itemize}
            \item Общий секретный ключ пропускается через алгоритм DeriveKey, который использует алгоритм Хеширования ключевых слов (HKDF) для вывода нескольких ключей шифрования из общего секрета.
            \item Один из полученных ключей может быть использован в качестве ключа шифрования для алгоритма ChaCha20.
        \end{itemize}

        \item Шифрование и расшифрование данных:
        \begin{itemize}
            \item После установления соединения каждая сторона может использовать ключ шифрования ChaCha20 для зашифрования и расшифрования данных, передаваемых по VPN-туннелю.
            \item ChaCha20 работает в режиме поточного шифрования, поэтому данные могут быть зашифрованы или расшифрованы блоками, по мере их передачи.
        \end{itemize}
    \end{enumerate}

    \subsection{Преимущества и недостатки алгоритма ChaCha20}

    \paragraph{Преимущества}
    \begin{itemize}
        \item Высокая производительность: ChaCha20 был разработан с учетом скорости работы на различных платформах, включая современные процессоры и мобильные устройства.
        Он обладает высокой скоростью шифрования и расшифрования данных.
        \item Безопасность: Алгоритм ChaCha20 предоставляет надежную защиту данных.
        Он был тщательно исследован и протестирован криптографическим сообществом, и на текущий момент не обнаружено серьезных уязвимостей.
        \item Простота реализации: Алгоритм ChaCha20 имеет относительно простую структуру и алгоритмические операции, что делает его реализацию и интеграцию в программное обеспечение относительно простыми.
        \item Малое количество памяти: ChaCha20 не требует больших объемов памяти для своей работы, что позволяет эффективно использовать его на устройствах с ограниченными ресурсами.
    \end{itemize}

    \paragraph{Недостатки}
    \begin{itemize}
        \item Отсутствие аутентификации: ChaCha20 является алгоритмом только для шифрования данных и не включает механизм аутентификации.
        Для обеспечения целостности и подлинности данных необходимо дополнительно использовать аутентификационные коды (например, HMAC).
        \item Небольшая история использования: В отличие от некоторых других алгоритмов шифрования, ChaCha20 имеет относительно небольшую историю использования в широком масштабе.
        Некоторые организации и стандарты могут предпочитать использовать более устоявшиеся алгоритмы с длительной историей проверки.
    \end{itemize}

    \newpage


    \newpage


    \section{Отличия между Salsa20 и ChaCha20}

    Salsa20 и ChaCha20 являются двумя схожими алгоритмами шифрования, разработанными Дэниэлем Дж. Бернштейном. Вот некоторые отличия между ними:

    \subsection{Количество раундов}

    Одно из основных отличий между Salsa20 и ChaCha20 заключается в количестве раундов операций.
    Salsa20 выполняет 8 или 12 раундов (в зависимости от версии), в то время как ChaCha20 выполняет 20 раундов.
    Большее количество раундов в ChaCha20 способствует повышенной безопасности и сопротивляемости различным атакам, но также может немного снижать производительность.

    \subsection{Структура и распределение операций}

    Salsa20 и ChaCha20 имеют немного различную структуру и распределение раундовых операций.
    В Salsa20 используется один основной цикл, состоящий из нескольких итераций смешивания (quarter round), а затем выполняются финальные операции для получения псевдослучайного потока байтов. В то время как ChaCha20 имеет свою собственную улучшенную структуру с более равномерным распределением операций между раундами.

    \subsection{Поддержка в различных платформах}

    ChaCha20 обычно предпочтительнее Salsa20 для новых приложений и протоколов, так как он более широко поддерживается в различных платформах, включая аппаратные ускорители и программные библиотеки. Это делает ChaCha20 более гибким в выборе платформы и обеспечивает лучшую совместимость.

    В целом, Salsa20 и ChaCha20 являются близкими родственниками и имеют много общих характеристик.
    Однако, различия в количестве раундов, структуре и поддержке платформ делают ChaCha20 более привлекательным для многих современных приложений и протоколов, где важны как безопасность, так и производительность.

    \newpage


    \section{Список литературы}

    \begin{enumerate}
        \item Миф про «мобильный» CHACHA20 \url{https://habr.com/ru/articles/545042/}.

        \item Реализация ChaCha20 на языке "C" из GitHub \url{https://github.com/marcizhu/ChaCha20}.

        \item Salsa20 на википедии \url{https://ru.wikipedia.org/wiki/Salsa20}

    \end{enumerate}

\end{document}
